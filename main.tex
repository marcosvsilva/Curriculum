%---------------------------------------------------------------------------------------------
%	INITIAL DOCUMENT CONFIGURATIONS
%---------------------------------------------------------------------------------------------
\documentclass[12pt, oneside, openany, a4paper, english, brazil]{abntex2}

\usepackage{cmap}				
\usepackage{lmodern}			
\usepackage[T1]{fontenc}
\usepackage[utf8]{inputenc}		
\usepackage{indentfirst}		
\usepackage{color}				
\usepackage{graphicx}			
\usepackage{multicol}
\usepackage{multirow}
\usepackage{lipsum}			
\usepackage[brazilian,hyperpageref]{backref}
\usepackage[alf]{abntex2cite}	
\usepackage{subfig} 
\usepackage{titlesec}

\titleformat{\section}{\Large\scshape\raggedright}{}{0em}{}[\titlerule] 
\titlespacing{\section}{0pt}{3pt}{3pt} 

%---------------------------------------------------------------------------------------------
% INIT DOCUMENT
%---------------------------------------------------------------------------------------------

\begin{document}

% Removes page numbering
\pagestyle{empty} 

%---------------------------------------------------------------------------------------------
%	NAME AND CONTACT INFORMATION
%---------------------------------------------------------------------------------------------

\begin{center}
    \par{
        \raggedcolumns{\Huge Marcos Vinicius Ribeiro Silva}           
    \par}
\end{center}

\section{Dados pessoais}

\begin{tabular}{rl}
	\textsc{Data de Nascimento:} & 06/11/1992\\
	\textsc{Celular:} & (62) 9 8156-1292\\
	\textsc{Email:} & \href{mailto:marcos.v.silva@live.com}{marcos.v.silva@live.com} \\
	\textsc{Portfólio:} & \href{https://github.com/marcosvsilva}{marcosvsilva} \\
	\textsc{Lattes:} & \href{ http://lattes.cnpq.br/6930019751033452}{Marcos Vinicius} \\
	\textsc{Objetivo:} & Engenharia de Software e Ciências de Dados \\
\end{tabular}

%---------------------------------------------------------------------------------------------
%	WORK EXPERIENCE 
%---------------------------------------------------------------------------------------------

\section{Formação Acadêmica}

\begin{tabular}{p{2.2cm}|p{12cm}}
    \emph{Mestrado}
    & \emph{em Ciência da Computação} \\
    & \emph{Instituto de Informática da Universidade Federal de Goiás} \\
    & \emph{Linha de sistemas inteligentes e suas aplicações} \\
    & Defesa prevista para: 1º semestre de 2021 \\
\end{tabular}

\begin{tabular}{p{2.2cm}|p{12cm}}
    \emph{Graduação}
    & \emph{em Engenharia de Software} \\
    & \emph{Instituto de Informática da Universidade Federal de Goiás} \\
    & Término: 2018\\
\end{tabular}

%---------------------------------------------------------------------------------------------
%	EDUCATION
%---------------------------------------------------------------------------------------------

\section{Experiência Profissionais e Desenvolvimento}

\begin{tabular}{p{3.5cm}p{11cm}}
    \textsc{Fevereiro 2019} - & Pesquisador \\
    \textsc{Atual} & Universidade Federal de Goiás \\
    \textsc{} & Bolsista CAPES (Coordenação de Aperfeiçoamento de Pessoal de Nível Superior) para pesquisa na área de inteligência artificial na linha de métodos inteligentes por imagem com a proposta de um novo método de diagnóstico de transtornos neurais em recém-nascidos através de imagens do exame pupilometria. O novo método utiliza uma rede neural artificial para rastreamento e localização pupilar com proposito de extrair as características pupilares imutáveis de pacientes para enfim propor um modelo de classificação capaz de realizar tal diagnóstico. Trabalho realizado com orientação do prof. \href{http://lattes.cnpq.br/6776569904919279}{Dr. Celso Gonçalves Camilo Junior}
\end{tabular}

\begin{tabular}{p{3.5cm}p{11cm}}
    \textsc{Novembro 2017} - & Desenvolvedor de Software \\
    \textsc{Fevereiro 2019} & Siagri Sistemas de Gestão LTDA \\
    \textsc{} & Desenvolvimento de novas funcionalidades e aplicações em linguagem Delphi para ERP de gestão de agribusiness, trabalho de relacionamento do escopo e entregas contínuas do scrum, correção de falhas imediatas e manutenção corretiva e preventiva de sistemas, implementação de camada de persistência de dados em linguagem PL-SQL utilizando banco de dados relacional Oracle e Firebird
\end{tabular}

\begin{tabular}{p{3.5cm}p{11cm}}
    \textsc{Março 2017} - & Desenvolvedor de Software \\
    \textsc{Setembro 2017} & SIACON Consultoria em Software LTDA ME \\ 
    \textsc{} & Solução de bugs e falhas, análise e implementação de melhorias, estruturação, modelagem e criação de serviços, ferramentas e relatórios em linguagem Delphi e C\#, manutenção e aprimoramento da camada de persistência em linguagem SQL
\end{tabular}                                    

\begin{tabular}{p{3.5cm}p{11cm}}
    \textsc{Abril 2013} - &  Bolsista de Iniciação Científica \\
    \textsc{Outubro 2015} & Universidade Federal de Goiás \\
    \textsc{} & Professor titular do programa de treinamento para a Olimpíada Brasileira de Informática da Universidade Federal de Goiás dedicada a estudantes do ensino fundamental e médio, com base em fundamentos básicos da programação de computadores na linguagem C e C++
\end{tabular}

\begin{tabular}{p{3.5cm}p{11cm}}
    \textsc{Setembro 2013} - & Desenvolvedor de Software \\
    \textsc{Setembro 2016} & Jave Informática LTDA \\ 
    \textsc{} & Manutenção e inovação de plataformas em Linguagem Delphi com utilização de componentes próprios e de terceiros com conexão direta e indireta a banco de dados relacionais utilizando linguagem SQL
\end{tabular}

%---------------------------------------------------------------------------------------------
%	COMPUTER SKILLS 
%---------------------------------------------------------------------------------------------

\section{Qualificações}

\begin{tabular}{p{4.5cm}p{10cm}}
    \textsc{I. A.:} & Aprendizado de Máquina, Redes Neurais, Mineração de Dados, Ciência de Dados, Métodos Estastísticos \\
    \textsc{Conhecimentos:} & Python, C, C++, Delphi, SQL, MYSQL, PL-SQL, C\#, Java, Android, \LaTeX \\
    \textsc{Banco de Dados:} & SQL Server, Oracle, MySQL, SQLite, PostgreSQL, MongoDB, Firbird \\
    \textsc{Ambiente Integrado:} & Anaconda, PyPi, PyCharm, CodeGear RAD Studio, IntelliJ IDEA, Jupyter Notebook, Android Studio, SQL Server Management Studio, Eclipse, Visual Studio \\
    \textsc{Versionamento:} & Git, GitHub, Microsoft Visual SourceSafe \\
    \textsc{Métodos Ágeis:} & Scrum e Programação Extrema ( XP ) \\
    \textsc{Design Patterns:} & Factory Method, Iterator \\
    \textsc{Linguagens:} & Inglês Avançado e Espanhol Básico \\
\end{tabular}

%---------------------------------------------------------------------------------------------
%	COURSES
%---------------------------------------------------------------------------------------------

\section{Cursos e Certificações}

\begin{tabular}{p{3.2cm}|p{11cm}}
    \emph{Curso online}
    & \emph{Python Para Análise de Dados} \\
    & \emph{Portal Data Science Academy} \\
    & \emph{Curso de Python do inicial ao avançado voltado a análise de dados e construção de aplicações de inteligência artificial;}
\end{tabular}

\begin{tabular}{p{3.2cm}|p{11cm}}
    \emph{Curso online}
    & \emph{Deep Learning I e II} \\
    & \emph{Portal Data Science Academy} \\
    & \emph{Curso que estuda as Redes Neurais Artificiais, Perceptrons de Camada Única, Perceptrons de Múltiplas Camadas e Redes Neurais Convolucionais, Redes Neurais Recorrentes, Redes Neurais Recursivas, Mapas Organizacionais, Memory Networks e Redes RBF, Generative Adversarial Networks, Autoencoders e Deep Reinforcement Learning.}
\end{tabular}

%---------------------------------------------------------------------------------------------
%	SCHOLARSHIPS AND ADDITIONAL INFO
%---------------------------------------------------------------------------------------------

\section{Atividades complementares}

\begin{tabular}{rl}
    2011 & Sociedade Brasileira para o Progresso da Ciência. 2011 \\
    2014 & Google Developers Group (GDG) DevFest Goiânia. 2014 \\
\end{tabular}

%---------------------------------------------------------------------------------------------
% CLOSED DOCUMENT
%---------------------------------------------------------------------------------------------

\end{document}